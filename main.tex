% -- Encoding UTF-8 without BOM
% -- XeLaTeX => PDF (BIBER)

\documentclass[hidelinks]{cv-style}          % Add 'print' as an option into the square bracket to remove colours from this template for printing. 
                                    % Add 'espanol' as an option into the square bracket to change the date format of the Last Updated Text

\sethyphenation[variant=british]{english}{} % Add words between the {} to avoid them to be cut 

\begin{document}

\header{Jonas }{Pfeiffer }           % Your name
\lastupdated

%----------------------------------------------------------------------------------------
%	SIDEBAR SECTION  -- In the aside, each new line forces a line break
%----------------------------------------------------------------------------------------
\begin{aside} % In the aside, each new line forces a line break

\includegraphics[width=\textwidth]{jonas_image.jpeg}

\section{contact}
\href{https://www.google.de/maps/place/Karl-Metz-Straße+5,+69115+Heidelberg/}{Karl-Metz-Str. 5}
\href{https://www.google.de/maps/place/Karl-Metz-Straße+5,+69115+Heidelberg/}{69115 Heidelberg}
\href{https://www.google.de/maps/place/Karl-Metz-Straße+5,+69115+Heidelberg/}{Germany}
~
* 29.11.1990
~
+49 (170) 909 86 47
\href{mailto:Jonas.Pfeiffer90@gmail.com}{Jonas.Pfeiffer90\\@gmail.com}
\href{https://github.com/JoPfeiff}{\emph{GitHub} JoPfeiff}
\href{https://www.linkedin.com/in/jonas-pfeiffer/}{\emph{LinkedIn} jonas-pfeiffer}
\href{https://scholar.google.com/citations?user=gGB0L4kAAAAJ}{\emph{GoogleScholar}}
\href{https://dblp.org/pers/hd/p/Pfeiffer:Jonas}{\emph{dblp}}
%\href{http://www.smith.com}{http://www.smith.com}
%\href{http://facebook.com/johnsmith}{fb://jsmith}
\section{languages}
\textbf{german} mother tongue
\textbf{english} fluent
\section{programming}
{\color{red} $\varheartsuit$} \textbf{Python}
Java, SQL, R, C
\section{frameworks}
\textbf{TensorFlow}, \textbf{PyTorch}, NumPy, pandas, matplotlib, MongoDB, Spark, BigQuery, Tableau
\end{aside}

%----------------------------------------------------------------------------------------
%	WORK EXPERIENCE SECTION
%----------------------------------------------------------------------------------------

\section{education}

\begin{entrylist}

%------------------------------------------------
\entry
{2019--now}
{PhD {\normalfont in Computer Science \emph{(3 Months)}}}
{\href{https://www.informatik.tu-darmstadt.de/ukp/ukp_home/index.en.jsp}{UKP - TU Darmstadt}}
{\textit{Natural Language Processing} \\ 
\textit{Research Focus:} Semi-Supervised Learning, Transfer Learning, Multi-Modal Learning, Interpretability of Black-Box Neural Architectures. 
}

%------------------------------------------------
\entry
{2015--2018}
{Master {\normalfont of Science \emph{(2.5 Years)}}}
{\href{https://www.wim.uni-mannheim.de/de/fakultaet/}{University of Mannheim}}
{\emph{Business Informatics - Data and Web Science} \\ 
\emph{Master Thesis} A Neural Network approach to Document Ranking and Query Refinement in Pharmacogenomics (Grade 1.3) \\
Strong focus on Machine Learning with courses like Data Mining 1 \& 2, Large Scale Data Management and Web Mining\\
\emph{Grade} 1.4}

%------------------------------------------------

\entry
{2017}
{Master {\normalfont of Science \emph{(5 Months Abroad)}}}
{\href{https://www.cics.umass.edu}{University of Massachusetts - Amherst}}
{\emph{Computer Science - Center for Data Science} \\ 
\emph{Courses} COMPSCI 683 - Artificial Intelligence, COMPSCI 589 Machine Learning, COMPSCI 691R Seminar Introduction to Recommender Systems \\
\emph{GPA} 4.0}

%------------------------------------------------

\entry
{2012--2015}
{Bachelor {\normalfont of Science \emph{(2.5 Years)}}}
{\href{http://www.wiwi.uni-augsburg.de/studium/studiengaenge/win/}{University of Augsburg}}
{\emph{Business Informatics - Operations \& Information Management} \\ 
\emph{Bachelor Thesis} Branch \& Price in Airline Scheduling (Grade 1.7)\\
Strong focus on Linear Programming \\
\emph{Grade} 1.9
 }

%------------------------------------------------

\end{entrylist}

%----------------------------------------------------------------------------------------
%	WORK EXPERIENCE SECTION
%----------------------------------------------------------------------------------------

\section{publications}
%
\begin{entrylist}

\entry
{2019}
{[UNDER REVIEW] What do Deep Networks Like to Read? Towards an Abstractive Method for Rationalizing Predictions}
{\href{https://www.informatik.tu-darmstadt.de/ukp/ukp_home/index.en.jsp}{UKP - TU Darmstadt}}
{\emph{ \textbf{\href{https://scholar.google.com/citations?user=gGB0L4kAAAAJ}{Jonas Pfeiffer}}, \href{https://ashkamath.github.io/}{Aishwarya Kamath}, \href{https://www.informatik.tu-darmstadt.de/ukp/ukp_home/staff_ukp/prof_dr_iryna_gurevych/index.en.jsp}{Iryna Gurevych},  \href{http://ruder.io/}{Sebastian Ruder}}\\
\href{https://www.aclweb.org}{ACL 2019} \\
}

\entry
{2019}
{[UNDER REVIEW] \\ Specializing Distributional Vectors of All Words for Lexical Entailment}
{\href{https://www.informatik.tu-darmstadt.de/ukp/ukp_home/index.en.jsp}{UKP - TU Darmstadt}}
{\emph{\href{https://ashkamath.github.io/}{Aishwarya Kamath*}, \textbf{\href{https://scholar.google.com/citations?user=gGB0L4kAAAAJ}{Jonas Pfeiffer*}}, \href{https://gogoglavas.wixsite.com/goran}{Goran Glavas}, \href{http://people.ds.cam.ac.uk/ep490/}{Edoardo Maria Ponti}, \href{https://sites.google.com/site/ivanvulic/shortbio}{Ivan Vulic}}\\
\href{https://www.aclweb.org}{ACL 2019} \\
}

\entry
{2018}
{\href{http://www.aclweb.org/anthology/W18-2310}{A Neural Autoencoder Approach for Document Ranking and Query Refinement in Pharmacogenomic Information Retrieval}}
{\href{https://www.wim.uni-mannheim.de/de/fakultaet/}{University of Mannheim}}
{\emph{\textbf{\href{https://scholar.google.com/citations?user=gGB0L4kAAAAJ}{Jonas Pfeiffer}}, \href{https://dblp.org/pers/hd/b/Broscheit:Samuel}{Samuel Broscheit}, \href{https://scholar.google.com/citations?user=s_GmFv0AAAAJ}{Rainer Gemulla}, \href{https://www.xing.com/profile/Mathias_Goeschl?sc_o=da980_e}{Mathias Göschl}}\\
\href{https://www.aclweb.org}{ACL 2018} - \href{https://aclweb.org/aclwiki/BioNLP_Workshop}{BioNLP Workshop} \\

}

\end{entrylist}

\newpage
\vspace{20cm}

\section{experience}

%\subsection{Full Time}

\begin{entrylist}

\entry
{2018}
{\href{https://www.spotify.com/}{Spotify}  {\normalfont \emph{(6 Months)}}}
{Stockholm, Sweden}
{\emph{Data Scientist - Netlight} \\
RnD Data Scientist at PPX-Insights - Analytics and Machine Learning\\
\emph{Tech} Python, SQL, BigQuery, Tensorflow, Tableau

}

\entry
{2018}
{\href{https://www.netlight.com/}{Netlight}  {\normalfont \emph{(6 Months)}}}
{Munich, Germany}
{\emph{Data Scientist  Machine Learning Consultant} \\
Data Scientist and Machine Learning Consultant focusing on NLP\\
\emph{Tech} Python, SQL, BigQuery, Tensorflow, Tableau

}

%------------------------------------------------

\entry
{2017--2018}
{\href{http://www.molecularhealth.com/}{Molecular Health GmbH}  {\normalfont \emph{(1 Year)}}}
{Heidelberg, Germany}
{\emph{Masterand - Machine Learning Researcher} \\
Developing a model to refine queries of biomedical curators. \\
\emph{Tech} Python, Tensorflow

}

%------------------------------------------------

%\end{entrylist}

%\subsection{Part Time}

%\begin{entrylist}

\entry
{2016}
{\href{https://home.kpmg.com/de/de/home/ueber-kpmg/offices/mannheim-1.html}{KPMG AG} {\normalfont \emph{(3 Months)}}}
{Mannheim, Germany}
{\emph{Advisory Data \& Analytics - Internship} 
\begin{itemize}
\item Reformulated SQL scripts into SAP - HANA
\item Support managers in their work (Presentations, Spreadsheets, etc.)
\end{itemize}
\emph{Tech} SQL, SAP-HANA

}

%------------------------------------------------

\entry
{2016}
{\href{http://dws.informatik.uni-mannheim.de/en/people/professors/profdrsimonepaoloponzetto/}{Chair of Information Systems III} {\normalfont \emph{(6 Months)}}}
{\href{https://www.wim.uni-mannheim.de/de/fakultaet/}{University of Mannheim}}
{\emph{Research Assistant - NLP}  Prof. Ponzetto\\
Project on Entity Recognition - Training samples creation\\
\emph{Tech} Java, Python

}

%------------------------------------------------





%------------------------------------------------



%------------------------------------------------



\end{entrylist}

\begin{entrylist}


\entry
{2015--2016}
{\href{https://stolletz.bwl.uni-mannheim.de/home}{Chair of Production Management} {\normalfont \emph{(7 Months)}}}
{\href{https://www.wim.uni-mannheim.de/de/fakultaet/}{University of Mannheim}}
{\emph{Research Assistant}  Prof. Stolletz\\
Advancement of a program for calculating different queuing theory formulas\\
\emph{Tech} Java

}




\entry
{2014--2015}
{\href{http://www.wiwi.uni-augsburg.de/bwl/tuma/}{Chair of Production \& Supply Chain Management} {\normalfont \emph{(1 Year)}}}
{\href{http://www.wiwi.uni-augsburg.de/studium/studiengaenge/win/}{University of Augsburg}}
{\emph{Research Assistant}  Prof. Tuma \\
%\begin{itemize}
%\item SAP Training Manager – “SAP Fallstudienkurs” 
%\item SAP Tutor - SAP Terp10
%\end{itemize}
\emph{Tech} SAP-GUI
}




%------------------------------------------------
\entry
{2012--2015}
{\href{http://www.munich-enterprise.com}{munich enterprise software GmbH} {\normalfont \emph{(3 Years)}}}
{Garching near Munich, Germany}
{\emph{Junior SAP Consultant}\\
Project Employee at BMW AG - ITZ \\
%Development, Implementation and Go-Live in Munich\\
\emph{Tech} ABAP
}
%\entry
%{2012--2014}
%{\href{http://www.munich-enterprise.com}{munich enterprise software GmbH} {\normalfont \emph{(1.5 Years)}}}
%{Garching near Munich, Germany}
%{\emph{Working Student}\\
%%SAP Support, EDP administration, backups
%}

%------------------------------------------------



%----------------------------------------------------------------------------------------
%	AWARDS SECTION
%----------------------------------------------------------------------------------------



\end{entrylist}


% \section{data science projects}

% %------------------------------------------------

% \begin{entrylist}
% \entry
% {2017}
% {A NN approach to DR and QR in Pharmacogenomics}
% {\href{https://www.wim.uni-mannheim.de/de/fakultaet/}{University of Mannheim}}
% {\emph{Master Thesis}\\
% Developing a model to refine queries of biomedical curators. \\
% Based on an input gene we want to automatically add entities that have a semantical relationship in the retrieved papers.\\
% Steps to build the model:
% \begin{enumerate}
% \item Learn Word Embeddings using Word2Vec
% \item Document Encoding using Seq2Seq approach
% \begin{itemize}
% \item  \emph{Encoder} Recurrent Neural Network (RNN) \emph{Decoder} RNN
% \item  \emph{Encoder} Convolutional Neural Network (CNN) \emph{Decoder} RNN
% \end{itemize}
% \item Query refinement - Initialized RNN
% \end{enumerate}
% \emph{Tech} Python, Tensorflow, Gensim, Spark, MongoDB \\ 
% (\href{https://github.com/JoPfeiff/seq2seq_dCNN}{https://github.com/JoPfeiff/seq2seq\_dCNN})
% }


% %------------------------------------------------

% \entry
% {2017}
% {Rhythm Classification of ECG Recordings}
% {\href{https://www.cics.umass.edu}{University of Massachusetts - Amherst}}
% {\emph{Machine Learning Project}\\
% Classification of ECG records into their rhythm types. \\
% \href{https://www.physionet.org/challenge/2017/}{https://www.physionet.org/challenge/2017/} \\
% We have developed a novel approach to peak detection, and compared many classification algorithms using gridsearch and crossvalidation to find the best model for the task.\\
% \emph{Tech} Python \\
% (\href{https://github.com/JoPfeiff/rhythm-classification}{https://github.com/JoPfeiff/rhythm-classification})
% }

% \entry
% {2016}
% {Transfermarkt - Graph Analysis}
% {\href{https://www.wim.uni-mannheim.de/de/fakultaet/}{University of Mannheim}}
% {\emph{Web Mining Project}\\
% \href{https://www.transfermarkt.de/}{https://www.transfermarkt.de/} \\
% Graph representation of football transfers (Nodes = Clubs, Edges = Transfers).  We analyzed graveyard clubs, gold lower level clubs, and detected club communities. We built a crawler, preprocessed the data (inconsistencies in the DB of transfermarkt), used standard metrics like reversed-pagerank, but also developed a new specified ranking measure. \\
% \emph{Tech} Python
% }

% %\entry
% %{2016}
% %{Data Mining Cup}
% %{\href{https://www.wim.uni-mannheim.de/de/fakultaet/}{University of Mannheim}}
% %{\emph{Data Mining 2 Project}}
% %
% %\entry
% %{2015}
% %{IMDb - Movie Rating Prediction}
% %{\href{https://www.wim.uni-mannheim.de/de/fakultaet/}{University of Mannheim}}
% %{\emph{Data Mining 1 Project}}


% \end{entrylist}

\section{scholarships}

\begin{entrylist}



%------------------------------------------------

\entry
{2017}
{Baden-Württemberg (Germany) Scholarship}
{\href{https://www.wim.uni-mannheim.de/de/fakultaet/}{University of Mannheim}}


%------------------------------------------------

\end{entrylist}

%----------------------------------------------------------------------------------------
%	COMMUNICATION SKILLS SECTION
%----------------------------------------------------------------------------------------

%\section{communication skills}

%\begin{entrylist}

%------------------------------------------------

%\entry
%{2011}
%{Oral Presentation}
%{California Business Conference}
%{Presented the research I conducted for my Masters of Commerce degree.}

%------------------------------------------------

%\entry
%{2010}
%{Poster}
%{Annual Business Conference, Oregon}
%{As part of the course work for BUS320, I created a poster analyzing several local businesses and presented this at a conference.}

%------------------------------------------------

%\end{entrylist}

%----------------------------------------------------------------------------------------
%	INTERESTS SECTION
%----------------------------------------------------------------------------------------

\section{interests}

\textbf{professional:} NLP, deep learning, machine learning, data science, python \\
\textbf{personal:} latin dancing, traveling 

%----------------------------------------------------------------------------------------
%	PUBLICATIONS SECTION
%----------------------------------------------------------------------------------------

%\section{publications}

%\printbibsection{article}{article in peer-reviewed journal} % Print all articles from the bibliography

%\printbibsection{book}{books} % Print all books from the bibliography

%\begin{refsection} % This is a custom heading for those references marked as "inproceedings" but not containing "keyword=france"
%\nocite{*}
%\printbibliography[sorting=chronological, type=inproceedings, title={international peer-reviewed conferences/proceedings}, notkeyword={france}, heading=bibheading]
%\end{refsection}

%\begin{refsection} % This is a custom heading for those references marked as "inproceedings" and containing "keyword=france"
%\nocite{*}
%\printbibliography[sorting=chronological, type=inproceedings, title={local peer-reviewed conferences/proceedings}, keyword={france}, heading=bibheading]
%\end{refsection}

%\printbibsection{misc}{other publications} % Print all miscellaneous entries from the bibliography

%\printbibsection{report}{research reports} % Print all research reports from the bibliography

%---------------
\end{document}